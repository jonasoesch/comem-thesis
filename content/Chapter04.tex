\chapter{Marketing Mix}\label{marketing-mix}

\section{Product}\label{product}

\subsection{Core Product}\label{core-product}

The problem of the clients is that there is not enough going on at
comem+. This affects students motivation, promotion and general school
culture in a negative way. The product tries to solve this problem on
different levels. It unites the creative potential of students and gives
the department a voice and with this a stronger identity.

\subsection{Actual Product}\label{actual-product}

The actual product is the podcast. It is the means to the end of the
core product and serves its purpose. A podcast is a short (2 minutes) to
medium (60 minutes) length audio programme which is delivered through
the internet. The contents may vary greatly but are in line with
students interests.

\begin{longtable}[c]{@{}lll@{}}
\toprule\addlinespace
Orientation & Profit driver & Western European timeframe
\\\addlinespace
\midrule\endhead
Production & Production methods & until the 1950s
\\\addlinespace
Product & Quality of the product & until the 1960s
\\\addlinespace
Selling & Selling methods & 1950s and 1960s
\\\addlinespace
Marketing & Needs of customers & 1970s to the present day
\\\addlinespace
Holistic Marketing & Everything matters & 21st century
\\\addlinespace
\bottomrule
\end{longtable}

\begin{Shaded}
\begin{Highlighting}[numbers=left,,]
\KeywordTok{public} \KeywordTok{class} \NormalTok{Point2D \{}
  
  \DataTypeTok{double} \NormalTok{x;}
  \DataTypeTok{double} \NormalTok{y;}
  
  \KeywordTok{public} \NormalTok{Point2D(}\DataTypeTok{double} \NormalTok{y, }\DataTypeTok{double} \NormalTok{x) \{}
   \KeywordTok{this}\NormalTok{.}\FunctionTok{x} \NormalTok{= x;}
   \KeywordTok{this}\NormalTok{.}\FunctionTok{y} \NormalTok{= y;}
  \NormalTok{\}}

  \KeywordTok{public} \DataTypeTok{static} \DataTypeTok{double} \FunctionTok{rendDistance}\NormalTok{(Point2D pt1, Point2D pt2) \{}
    
    \DataTypeTok{double} \NormalTok{distance = }\DecValTok{0}\NormalTok{;}
    \KeywordTok{if}\NormalTok{( (pt1 != }\KeywordTok{null}\NormalTok{) && (pt2 != }\KeywordTok{null}\NormalTok{) ) \{}

      \DataTypeTok{double} \NormalTok{distX = pt1.}\FunctionTok{rendX}\NormalTok{() - pt2.}\FunctionTok{rendX}\NormalTok{();}
      \DataTypeTok{double} \NormalTok{distY = pt1.}\FunctionTok{rendY}\NormalTok{() - pt2.}\FunctionTok{rendY}\NormalTok{();}
      

      \NormalTok{distance = Math.}\FunctionTok{sqrt}\NormalTok{( Math.}\FunctionTok{pow}\NormalTok{(distX, }\DecValTok{2}\NormalTok{) + Math.}\FunctionTok{pow}\NormalTok{(distY, }\DecValTok{2}\NormalTok{));}
    \NormalTok{\}}
    
    \KeywordTok{return} \NormalTok{distance;}
  \NormalTok{\}}

  
  \KeywordTok{public} \DataTypeTok{double} \FunctionTok{rendDistanceA}\NormalTok{(Point2D pt2) \{}
      \KeywordTok{return} \KeywordTok{this}\NormalTok{.}\FunctionTok{rendDistance}\NormalTok{(}\KeywordTok{this}\NormalTok{, pt2);}
  \NormalTok{\}}
  \KeywordTok{public} \DataTypeTok{double} \FunctionTok{rendX}\NormalTok{() \{}
    \KeywordTok{return} \KeywordTok{this}\NormalTok{.}\FunctionTok{x}\NormalTok{;}
  \NormalTok{\}}

  \KeywordTok{public} \DataTypeTok{double} \FunctionTok{rendY}\NormalTok{() \{}
    \KeywordTok{return} \KeywordTok{this}\NormalTok{.}\FunctionTok{y}\NormalTok{;}
  \NormalTok{\}}
\NormalTok{\}}
\end{Highlighting}
\end{Shaded}

\subsection{Augmented Product}\label{augmented-product}

As the podcasts are produced in a secondary language, students enjoy the
benefit of improved language skills. This language learning can be
considered a service which adds value to the product. As there are
students with different mother-tongues working together they profit from
each others knowledge. Additionally there is always a language professor
available for further help.

Additional values may be created by partnership with other organizations
like the BALEINEV committee.

\section{Price}\label{price}

No money is demanded in exchange for the product. It demands however a
certain amount of attention and time of the customer. By providing
products and services which demand different levels of attention and
engagement from the clients, the project tries to penetrate the market
as much as possible.

Starting on the low end there are short shows on a variety of topics
(like cooking or school culture) which range from two to ten minutes.

Upwards there are longer, more refined shows which treat topics more
deeply.

For clients who want to get involved, the project offers ways to get in
touch with the creators and discuss with them.

The highest level of possible engagement is the participation in the
project. As it is open to about everyone, very motivated people are
given the possibility to significantly influence the product.

In comparison to the concurrence we shoot at the same time lower and
also higher when it comes to the demanded attention. Most podcast are
longer than ten minutes. But also the maximum level of engagement is
usually some kind of talk-back.

\section{Promotion}\label{promotion}

The main promotional element is word of mouth promotion. As many
students come in contact with the project during their studies they
automatically tell others about it. This is a free and very powerful
promotional element and should work very well because the core market is
small.

Social media helps with word of mouth promotion as people can easily
share the project with their contacts or even discuss shows they liked
with the ones who have created them.

To reach our extended market the project should consider partnerships
with well-established enterprises and organizations with a certain reach
in our extended market. On example would be the association of Swiss
media engineers (VSMI/ASIM)\footnote{VSMI/ASIM. Association which joins
  graduated media engineers from Switzerland. http://www.vsmi-asim.ch}
which could help us to reach graduated media engineers.

\section{Place}\label{place}

The only distribution channel is the internet. This helps the project
deliver broadcast at a very low cost and with global reach. Indirect sub
channels are Soundcloud and iTunes. There is also a direct sub channel
which is the projects website. This diversification in sub channels
helps minimizing the risk of being cut off completely from the
listeners.

\section{Physical Evidence}\label{physical-evidence}

There is a very tangible part in the project, which are the podcasts.
But as there are also quite a few service aspects to the project, it
must consider some sort of physical evidence. For the language learning
part, participants need to maintain a record of their tasks,
achievements and problems. Based on this and the actual results they
will receive a mark to represent the quality of their language learning
efforts.

When it comes to the school marketing there is the projects website
which also feeds the official department blog and will hopefully revive
it a bit.

As for school culture, the project tries to provide a convivial
environment inside of the CMC by providing sofas and other
infrastructure for the students to dispose of.

\section{Partners}\label{partners}

The project is completely built on partnerships inside and outside of
the department. First of all, many of the participants are willing to
share their prior knowledge and help others out. Examples for this may
be some students helping others with the technical aspects of audio
recording but also with difficulties in a secondary language. The
participants therefore do also gain a lot from the project in terms of
knowledge exchange and experience.

A very fertile partnership with professors and collaborators should help
to push the project into the right direction as these partners bring in
their experience and their farsightedness. These partners may hope to
get access to the most motivated students and a very contemporary way of
teaching. The ones who are ready to go out of their comfort zone, may
also benefit from knowledge exchange.

Not to forget the school, which provides crucial financing and
infrastructure. The school partners with (or better supports) the
project as it may be a marketing instrument in the future.

To finish, there are also quite a few people who are willing to help out
the students by participating in a show on their request. This might be
an act of generosity, a general interest in the project or the school or
even for their own publicity.
